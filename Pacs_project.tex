\documentclass[]{article}
\usepackage{fancyhdr}


%opening
\title{APCS project, A.Y. 2019-2020\\Project proposal \#50:"Development of an interactive 2D application with  OpenFrameworks"}
\author{Antonio Pipita}

\begin{document}

\maketitle
\newpage
\tableofcontents
\newpage
\section{Introduction}
This project consists in developing an application that enables the user to interact with a 2D environment via video feed.\\
In particular, the user will control via hand motion a component of a simple game: the goal of the player is to destroy a brick wall by making a ball bounce into it, while avoiding to make the ball fall off the screen. In order to do so, the player will move a bar at the bottom of the playfield via motion control.\\
The video feed will be captured via a Microsoft Kinect, mod. 1414\\
\section{Application architectural overview}
This section introduces the application outlining its structure and the information flow.
\subsection{Design pattern}
This application has been designed following the Model-Controller-View pattern:
\begin{itemize}
	\item [Model]: contains the application data and manages its update. In this application it comprehends the classes play\_field, ball, bar and brick.\\
				It manages the movement of the ball on the playfield, its interactions with the bricks, the bar and the playfield edges. It also recives the updates on the bar position and angle, if the required action is possible. 
	\item [Controller]: manages the flow of information from the view to the controller. In this application it comprehends the class ofApp, which grabs input from the kinect device, re-elaborates it and pushes it to the model.
	\item [View]: manages the interaction with the user. In this application it comprehends the painter class, which represents on the device screen the status of the model, partially the ofApp class, since it directly draws the kinect's depth view on the screen, and the hardwar used by the user for the inputs, which comprises the kinect itself and the keyboard. 
\end{itemize}
The splitting of the view between the painter class and the ofApp class was done in order to keep the kinect interaction just in the ofApp class, thus keeping the code a bit more tidy.
\subsection{Information flow}
The information that the user want to transmit enters the application via the view (kinect and keyboard), is then interpretedd and restructured by the controller and passed to the model. The model applies the requiredd changes to the data, updates itself following the rules of ball movement and then send the new infto the view again, which will print on the screen the current state of the game and give the user new information.\\
\subsection{Information decoupling}
The information represented in the view is not the one present in the model, but its representation. Changes to the view ddo not affect the data.
\section{Classes overview}
Tis section contains the class diagram and a description of the attributes and methods of each class
\subsection{ofApp}
\subsection{Playfield}
\subsection{Bar, brick and ball}
\subseciton{Painter}
\section{Application workflow}
\section{Final remarks}  
\end{document}